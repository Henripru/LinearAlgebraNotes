\documentclass[12pt]{article}%you can add twocolumn as another argument to make into double column
\usepackage[margin=1in]{geometry} % for customizing page layout
\usepackage{graphicx}%package for adding images to document
\usepackage{amsthm, amsmath, amssymb}%packages for math 
\usepackage{ragged2e}%for cleaner look.
\usepackage[onehalfspacing]{setspace}
\usepackage[loose, nice]{units} %replace "nice" by "ugly" for units in upright fractions
 
\title{Computer Vision Mathematics: Homework 3}
\author{Henri Prudhomme}
\date{March 10, 2021}

\begin{document}
\maketitle

\section{Linear Algebra Solutions}

\begin{itemize}
\item 1.7 \# 19: 
A diagonal entry $s_{jj}$ of a symmetric matrix cannot be smaller than all the $\lambda$'s.
Suppose this were false, then $S-s_{jj}I$ would have positive eigenvalues. 

There are two ways we can come to this conclusion.

\begin{enumerate}
\item
\begin{itemize}
\item[] $S = Q\Lambda Q^T$
\item[] $S-s_{jj}I = Q\Lambda Q^T - s_{jj}I$
\item[] $S-s_{jj}I = Q\Lambda Q^T - Q(s_{jj}I)Q^T$
\item[] $S-s_{jj}I = Q(\Lambda -s_{jj}I)Q^T$
\item[] Let $\Lambda_2 = \Lambda -s_{jj}I$
\item[] $S = Q\Lambda_2 Q^T$
\end{itemize}
\item
\begin{itemize}
\item[] Equation 5 from 1.6: $(A + s I)x = \lambda x + s x = (\lambda + s)x$

Applying to this problem:$(S-s_{jj}I)x = \lambda x -s_{jj} x = (\lambda - s_{jj}) x$
\end{itemize}
\end{enumerate}

Therefore $$\lambda_i - s_{jj} > 0 |_{i=0...n}$$
If this is the case, then the matrix is Symmetric Positive Definite.

However, from 1.7 \# 18, we know that a symmetric positive definite matrix cannot have a $0$ along its diagonal. Therefore, the first statement must be true.

\item 
\end{itemize}

\end{document}